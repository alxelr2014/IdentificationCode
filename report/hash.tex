\section{Generalizations of 3-step algorithm}
In the 3-step algorithm, we produce two random prime numbers and then compute a function of the message based on these two primes. In general, this is viewed as producing a local randomness \(j\) and computing a message specific tag function \(T_i\) with input \(j\), \(\func{T_i}{j}\) \cite{derebeyoglu}. When working with a noiseless, the tagging function determines the second kind error probability. For example, the probability that the receiver identifies \(\hat{i}\) when \(i\) is sent is calaculated as 
\begin{equation*}
    \func{P_{e,2}}{i,\hat{i}} = \prob{\func{T_i}{j} = \func{T_{\hat{i}}}{j}}.
\end{equation*} 
When \(j\) is chosen uniformly, this probability can be viewed as the probability of the collision of the hash function \(\func{h_j}{x} = \func{T_x}{j}\). The families of hash function for which this probability of collision is bounded are called \textit{almost universal}. A formal definition is given as follows in \cite{stinson}.
\begin{definition}
    Let \(H = \set{h : X \to Y}\) be a family of hash functions from \(X\) to \(Y\) and let \(\epsilon\) be a positive real number. \(H\) is said to be \(\epsilon\)-almost universal if for all distinct \(x_1, x_2 \in X\) 
    \begin{equation*}
        \set<h \in H>{\func{h}{x_1} = \func{h}{x_2}} \leq \epsilon \abs{H}
    \end{equation*}
    If \(h \gets H\) uniformly, 
    \begin{equation*}
        \prob{\func{h}{x_1} = \func{h}{x_2}} \leq \epsilon
    \end{equation*}
\end{definition}

Consider the following identification scheme for noiseless channels that uses almost universal functions.

\begin{definition}\label{def:hash3step}
    Let \(\mathcal{M} = \set{1 , \dots,  M}\) and \(H = \set{h_{a} : X \to Y}_{a \in I}\) be a family of \(\epsilon\)-almost universal hash functions indexed by the set \(I\). A round of communication round works as follows.
    \begin{enumerate}
        \item The sender chooses \(a \gets I\) uniformly. Then, he sends the code \((a, \func{h_a}{m})\) to the receiver.
        \item Upon receiving the code \((a, \func{h_a}{m})\), the receiver calculates \(\func{h_a}{\hat{m}}\) and identifies \(\hat{m}\) whenever \(\func{h_a}{m} = \func{h_a}{\hat{m}}\). 
    \end{enumerate}  
\end{definition}
The blocklength of the identification scheme in \Cref{def:hash3step} is 
\begin{equation*}
    n = \lg \abs{I} + \lg \abs{Y}
\end{equation*}
The number of random bits required is 
\begin{equation*}
    r = \lg \abs{I}
\end{equation*}
And the second kind error probability is given by 
\begin{equation*}
    \condProb{\func{h_a}{m} = \func{h_a}{\hat{m}}}{m \neq \hat{m}} \leq \epsilon
\end{equation*}

\subsection{Explicit Construction}

%  We now give an explicit constuction of the \Cref{def:hash3step}. Let \(\func{A}{n} = \set{1,2, \dots, 2^n }\), \(\func{B}{n} = \set{1,2, \dots ,n }\), \(K = \set{\frac{n}{2} , \dots, n}\), and \(\func{H}{n} = \set<h_k : A \to B>{k \in K}\) given as follows.
%  \begin{equation}
%     \func{h_k}{a} = \squareBracket{a \mod k} + 1
%  \end{equation}
%  \begin{lemma}\label{lmm:hashuniv}
%     The class \(\func{H}{n}\) described above is \(\frac{4}{n}\)-almost universal.
%  \end{lemma}

%  \begin{proof}
%     Suppose \(x , y \in A\) are distinct. 
%     \begin{equation*}
%         \prob{\func{h_k}{x} = \func{h_k}{y}} = \dfrac{2}{b} \abs{\set{k \in K : k \mid  \abs{x-y}}}
%     \end{equation*}
%     Let \(z = \abs{x - y}\) and assume that a key \(k\) satisfies \(k \mid z\). Since \(z \neq 0\), then it must necessarily be greater than \(k \geq \frac{n}{2}\). Moreover, \(z \leq n\) and hence, if \(qk = z\) for some positive integer \(q\), it must be that \(q \in \set{1,2}\). Therefore, there are at most two keys that satisfy \(k \mid z\). Consquently,
%     \begin{equation*}
%         \prob{\func{h_k}{x} = \func{h_k}{y}} \leq \dfrac{4}{b}
%     \end{equation*}
%  \end{proof}
%  Note that, \(H\) digests input exponentially, therefore, if we use \(H\) twice we can achieve double exponential compression. Firstly, consider the follwoing composition lemma.
%  \begin{lemma}\label{lmm:hashcomp}
%     Suppose \(H_1: A \to B\) and \(H_2: B \to C\) are \(\epsilon_1\) and \(\epsilon_2\)-almost universal, respectively. The hash famliy \(H = H_2 \circ H_1 : A \to C = \set<h_2 \circ h_1>{h_1 \in H_1, h_2 \in H_2}\) is \(\epsilon = \epsilon_1 + \epsilon_2\)-almost unviersal.
%  \end{lemma}

% \begin{proof}
%     Let \(x, y \in A\) be two distinct elements.
%     \begin{align}
%         \prob{\func{h_2}{\func{h_1}{x}} = \func{h_2}{\func{h_1}{y}}} 
%         &\leq \condProb{\func{h_2}{\func{h_1}{x}} = \func{h_2}{\func{h_1}{y}}}{\func{h_1}{x} \neq \func{h_1}{y}} +  \prob{\func{h_1}{x} = \func{h_1}{y}}\\
%         &\leq \epsilon_2 + \epsilon_1
%     \end{align}
% \end{proof}

% By \Cref{lmm:hashuniv} and \Cref{lmm:hashcomp}, the hash family \(\func{H^2}{n} = \func{H}{n} \circ \func{H}{2^n}\) is
% \begin{equation}
%     \epsilon = \frac{4}{n} + \frac{4}{2^n} \approx \frac{4}{n}
% \end{equation}
% almost universal which maps \(A = \set{0, 1, \dots, 2^{2^n}}\)