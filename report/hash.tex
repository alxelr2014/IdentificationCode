\section{Generalizations of 3-step algorithm}
In the 3-step algorithm, we produce two random prime numbers and then compute a function of the message based on these two primes. In general, this is viewed as producing a local randomness \(j\) and computing a message specific tag function \(T_i\) with input \(j\), \(\func{T_i}{j}\) \cite{derebeyoglu}. When working with a noiseless channel, the tagging function determines the second kind error probability. For example, the probability that the receiver identifies \(\hat{i}\) when \(i\) is sent is calaculated as 
\begin{equation}
    \func{P_{e,2}}{i,\hat{i}} = \prob{\func{T_i}{j} = \func{T_{\hat{i}}}{j}}.
\end{equation} 
When \(j\) is chosen uniformly, this probability can be viewed as the probability of the collision of the hash function \(\func{h_j}{x} = \func{T_x}{j}\). The families of hash functions for which this probability of collision is bounded are called \textit{almost universal}. A formal definition was first given in \cite{carter} which is as follows.
\begin{definition}
    Let \(H = \set{h : X \to Y}\) be a family of hash functions from \(X\) to \(Y\) and let \(\epsilon\) be a positive real number. \(H\) is said to be \(\epsilon\)-almost universal if for all distinct \(x_1, x_2 \in X\) 
    \begin{equation}
        \abs{\set<h \in H>{\func{h}{x_1} = \func{h}{x_2}}} \leq \epsilon \abs{H}
    \end{equation}
    If \(h \gets H\) uniformly, 
    \begin{equation}
        \prob{\func{h}{x_1} = \func{h}{x_2}} \leq \epsilon
    \end{equation}
\end{definition}

Consider the following identification scheme for noiseless channels that uses almost universal functions.

\begin{definition}\label{def:hash3step}
    Let \(\mathcal{M} = \set{1 , \dots,  M}\) and \(H = \set{h_{a} : X \to Y}_{a \in [I]}\) be a family of \(\epsilon\)-almost universal hash functions indexed by the set \([I] = \set{0, 1 , \dots , I - 1}\). A round of communication round works as follows.
    \begin{enumerate}
        \item The sender chooses \(a \gets [I]\) uniformly. Then, he sends the code \((a, \func{h_a}{m})\) to the receiver.
        \item Upon receiving the code \((a, \func{h_a}{m})\), the receiver calculates \(\func{h_a}{\hat{m}}\) and identifies \(\hat{m}\) whenever \(\func{h_a}{m} = \func{h_a}{\hat{m}}\). 
    \end{enumerate}  
\end{definition}
The blocklength of the identification scheme in \Cref{def:hash3step} is 
\begin{equation}
    n = \lg I + \lg \abs{Y}
\end{equation}
The number of random bits required is 
\begin{equation}
    r = \lg I
\end{equation}
And the second kind error probability is given by 
\begin{equation}
    \condProb{\func{h_a}{m} = \func{h_a}{\hat{m}}}{m \neq \hat{m}} \leq \epsilon
\end{equation}
Thus, we have shown the following statement.
\begin{theorem}
    If there exists a class of \(\epsilon\)-almost universal hash functions \(H = \set{h_a : \mathbb{F}_q^n \to \mathbb{F}_q}_{a \in [I]}\) then there exists a \((\lg I + \lg q, q^n,0, \epsilon )\) identification code for noiseless binary channel.
\end{theorem}

\subsection{Formulation of 3-step algorithm in terms of universal hash function}

 We now give an explicit constuction of the \Cref{def:hash3step} according to the \Cref{subsec:modified3step}. Let \(\func{A}{n} = \set{1,2, \dots, 2^n }\), \(\func{B}{n} = \set{1,2, \dots ,n^{\alpha} }\) for some \(\alpha > 1\), \(K = \set{2, 3, \dots, p_{\func{\pi}{n^{\alpha}}}}\), and define hash family \(\func{H}{n} = \set<h_k : A \to B>{k \in K}\) given as follows.
 \begin{equation}
    \func{h_p}{a} = \squareBracket{a \mod p} + 1
 \end{equation}
 \begin{lemma}\label{lmm:hashuniv}
    The class \(\func{H}{n}\) described above is \(\frac{\alpha}{n^{\alpha - 1}}\)-almost universal.
 \end{lemma}

 \begin{proof}
    Suppose \(x , y \in A\) are distinct. 
    \begin{align}
        \prob{\func{h_p}{x} = \func{h_p}{y}} &= \dfrac{1}{\func{\pi}{n^{\alpha}}} \abs{\set{p \in K : p \mid  \abs{x-y}}}\\
        & \leq \dfrac{\func{\omega}{x - y}}{\func{\pi}{n^{\alpha}}}\\
        & \approx \dfrac{\alpha}{n^{\alpha - 1}}
    \end{align}
 \end{proof}
 Note that, \(H\) digests input exponentially, therefore, if we use \(H\) twice we can achieve double exponential compression. Firstly, consider the follwoing composition lemma.
 \begin{lemma}\label{lmm:hashcomp}
    Suppose \(H_1: A \to B\) and \(H_2: B \to C\) are \(\epsilon_1\) and \(\epsilon_2\)-almost universal, respectively. The hash famliy \(H = H_2 \circ H_1 : A \to C = \set<h_2 \circ h_1>{h_1 \in H_1, h_2 \in H_2}\) is \(\epsilon =( \epsilon_1 + \epsilon_2)\)-almost unviersal. It is shown that \(H\) is \((\epsilon_1 + \epsilon_2 - \epsilon_1\epsilon_2)\)-almost universal \cite{bierbrauer}.
 \end{lemma}

\begin{proof}
    Let \(x, y \in A\) be two distinct elements.
    \begin{align}
        \prob{\func{h_2}{\func{h_1}{x}} = \func{h_2}{\func{h_1}{y}}} 
        &\leq \condProb{\func{h_2}{\func{h_1}{x}} = \func{h_2}{\func{h_1}{y}}}{\func{h_1}{x} \neq \func{h_1}{y}} +  \prob{\func{h_1}{x} = \func{h_1}{y}}\\
        &\leq \epsilon_2 + \epsilon_1
    \end{align}
\end{proof}

By \Cref{lmm:hashuniv} and \Cref{lmm:hashcomp}, the hash family \(\func{H^2}{n} = \func{H}{\alpha n} \circ \func{H}{2^n}\) is
\begin{equation}
    \epsilon = \frac{\alpha}{(\alpha n)^{\alpha - 1}} + \frac{\alpha}{2^{n \bracket{\alpha - 1}}} = \bigO{\dfrac{1}{n^{\alpha - 1}}}
\end{equation}
-almost universal which maps \(A = \set{0, 1, \dots, 2^{2^n}}\) to \(C = \set{1, \dots, (\alpha n)^{\alpha}}\).
\subsection{Current Directions}
One way to construct identification codes is by re-purposing error correcting codes \cite{verdu,derebeyoglu}. The methods described in \cite{verdu,derebeyoglu} use the error correction codes to construct a tag function. As a matter of fact, Bierbrauer \textit{et al} \cite{bierbrauer} has shown the equivalence between error correcting codes and almost universal hash function. 

\begin{theorem}\label{thm:hasheqid}
    If there is a \([n,k,d]_q\) code, then there exists a \((1- \frac{d}{n})\)-almost universal hash family \(H = \set{h_a : \mathbb{F}_q^k \to \mathbb{F}_q}_{a \in [n]}\) and conversely, if there is a \(\epsilon\)-almost universal  \(H = \set{h_a :  \mathbb{F}_q^k \to \mathbb{F}_q}_{a \in [n]}\) then there exists a \([n, k,n(1- \epsilon)]_q\) code.
\end{theorem}

This theorem gives us the framework needed to compare code-based identification schemes such as \cite{derebeyoglu}.

\begin{example}
    The 3-step algorithm is equivalent to almost universal hash family \(H^2\), as described above. The hash functions are indexed by
    \begin{equation}
        I = \set<(p_i,p_j)>{1 \leq i \leq \func{\pi}{2^{n\alpha}},\; 1 \leq j \leq (\alpha n)^{\alpha} }
    \end{equation}
    Instead of transmitting the indicies \((i,j)\) we transmit the actual key \((p_i,p_j)\). Thus, to transmit a key pair we need send \(n_1\) bits.
    \begin{align}
        n_1 &= \lg p_{\func{\pi}{2^{n\alpha}}} + \lg p_{\func{\pi}{(\alpha n)^{\alpha}}}\\
        &\sim \lg \func{\pi}{2^{n\alpha}} + \lg \ln \func{\pi}{2^{n\alpha}} + \lg \func{\pi}{(\alpha n)^{\alpha}} + \lg \ln \func{\pi}{(\alpha n)^{\alpha}}\\
        &= \bracket{1 + \littleO{1}} \lg \func{\pi}{2^{n\alpha}}\\
        &\sim \bracket{1 + \littleO{1}} \bracket{\lg {2^{n\alpha}} - \lg \ln 2^{n\alpha}}\\
        &= \bracket{1 + \littleO{1}} \alpha n
    \end{align}
    The output set is \(\set{1,\dots, (\alpha n)^{\alpha}}\) which requires \(\alpha \lg \alpha n\) bits to transmit. Therefore, the total blocklength is \(n' = \bracket{1 + \littleO{1}} \alpha n\). The input set is \(\set{1,\dots, 2^{2^n}}\) and the probability of collision is bounded by \(\bigO{\frac{1}{n^{\alpha - 1}}}\). As a result, we have an 
    \begin{equation}
        \bracket{\bracket{1 + \littleO{1}} \alpha n,\; 2^{2^n},\; 0,\; \bigO{\dfrac{1}{n^{\alpha - 1}}}}
    \end{equation}
    identification scheme.
\end{example}

\begin{example}
    The identification code in \cite{derebeyoglu} is constructed from a double Reed-Solomon code
\begin{equation}
    [q^{k+1}, k q^{k- \delta}, (q-k+1)(q^k - q^{k-\delta} +1)]_q
\end{equation}
    By \Cref{thm:hasheqid}, there exists an almost universal \(H = \set{h_a : \mathbb{F}_q^{kq^{k - \delta}} \to \mathbb{F}_q}_{a \in [q^{k+1}]}\) with the collision probability bounded by 
    \begin{equation}
        1 - \dfrac{(q-k+1)(q^k - q^{k-\delta} +1)}{q^{k+1}} \leq \dfrac{k-1}{q} + \dfrac{1}{q^{\delta}}
    \end{equation}
    Therefore, the resulting identification code is 
    \begin{equation}
        \bracket{(k+2)\lg q,\; q^{kq^{k - \delta}},\; 0 ,\; \dfrac{k-1}{q} + q^{-\delta}}
    \end{equation}
    % Suppose \(\set{(\func{q}{n},\func{k}{n},\func{\delta}{n})}\) characterizes a family of optimal capacity achieving code. Then, \(\func{q}{n}\) and \(\func{k}{n}\) must be increasing functions satisfying \cite{derebeyoglu}
    % \begin{align}
    %     \lim_{n \to \infty} \dfrac{\func{k}{n}}{\func{q}{n}} &= 0 & \lim_{n \to \infty} \dfrac{\func{\delta}{n}}{\func{k}{n}} &= 0 
    % \end{align}
\end{example}

