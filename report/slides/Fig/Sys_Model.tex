\tikzstyle{farbverlauf} = [ top color=white, bottom color=white!80!gray]
\tikzstyle{block1} = [draw,top color=white, middle color=white!80!orange, rectangle, rounded corners,
minimum height=2em, minimum width=2.5em]
\tikzstyle{block2} = [draw,top color=white, middle color=white!80!orange, rectangle, rounded corners,
minimum height=2em, minimum width=2.5em]
\tikzstyle{input} = [coordinate]
\tikzstyle{sum} = [draw, circle,inner sep=0pt, minimum size=5mm, thick, color=cyan]
\tikzstyle{arrow}=[draw,->]
\begin{tikzpicture}[auto, node distance=2cm,>=latex']
\node[] (M) {$i$};
\node[block1,right=.5cm of M] (enc) {Enc};
\node[sum, right=2cm of enc] (channel) {$+$};
% \node[below=.5mm of channel] (mod2) {$\text{\scriptsize mod 2}$};
\node[block2, right=2cm of channel] (dec) {Dec};
\node[below=.5cm of dec] (Target) {$\mathcal{S} = \{ j_{\text{1}}, \ldots, j_{\text{K}} \}$};
\node[right=.5cm of dec] (Output) {$\text{\small Yes/No}$};
\node[above=.7cm of channel] (noise) {$\textbf{Z}\sim \text{Ber($\epsilon$)}$};
\draw[->] (M) -- (enc);
\draw[->] (enc) --node[above]{$\textbf{u}_i$} (channel);
\draw[->] (noise) -- (channel);
\draw[->] (channel) --node[above]{$\textbf{Y}$} (dec);
\draw[->] (dec) -- (Output);
\draw[->] (Target) -- (dec);
\end{tikzpicture}
