\subsection{Prime Number Theorem}
\begin{theorem}[\cite{apostol}]\label{thm:pnt}
	Let \(\pi(x)\) denote the number of primes less than or equal to \(x \geq 1\). The prime number theorem states that 
	\begin{equation*}
		\lim_{x \to \infty} \dfrac{\pi(x) \ln x}{x} = 1
	\end{equation*}
	That is, \(\pi(x) \sim \dfrac{x}{\ln x}\). This implies that \(n\)-th prime \(p_n\) is asymptotically given by \(p_n \sim n \ln n\).
\end{theorem}

Moreover, we may frequently use these exact bounds on \(\pi(x)\).
\begin{lemma}[\cite{apostol}]\label{lmm:ineqpnt}
	For any \(n \geq 2\), 
	\begin{equation*}
		\dfrac{1}{6} \dfrac{n}{\ln n} \leq \pi(n) \leq 6 \dfrac{n}{\ln n}
	\end{equation*}
	and for any \(n \geq 1\),
	\begin{equation*}
		\dfrac{1}{6} n \ln n \leq p_n \leq 12 (n \ln n + n \ln \frac{12}{e})
	\end{equation*}
\end{lemma}