


\usepackage{tgpagella}
\usepackage[T1]{fontenc}
\usepackage{mdframed}

\setcounter{footnote}{0}

\let\svtikzpicture\tikzpicture
\def\tikzpicture{\noindent\svtikzpicture}
% \usepackage[center]{caption}
\usepackage[caption=false]{subfig}
\usepackage{caption}
\captionsetup{font=small}
\captionsetup[subfigure]{labelformat=empty}
% \usepackage{etex}
% \usepackage[thinc]{esdiff}
% \hyphenation{op-tical net-works semi-conduc-tor}
\usepackage{commath,amsmath,amssymb,amsfonts}
\usepackage{mathtools}
\usepackage{amsthm}
\usepackage{algorithmic}
\usepackage{pgfplots} 
\pgfplotsset{compat=1.15}
\usepackage{graphicx}
\usepackage{pgfgantt}
\usepackage{pdflscape}
\usepackage{pst-plot}
\usepackage{xfrac}
\usepackage{colortbl}
\usepackage{cancel}
\usepgfplotslibrary{fillbetween}
\usepackage{amssymb,bm}
\usepackage{float}
\usepackage{amsmath}
% \usepackage{hyperref}
\usepackage[draft=false]{hyperref}
\usepackage{multirow}
\usepackage{xcolor}
\usepackage{mathrsfs}
\usepackage{bbm}

\usepackage{cite} 

\usepackage{comment}



% \input{tikz-includes.tex}#
%Matlab
% \documentclass{beamer}
\usepackage{xcolor}
\usepackage{listings}
\lstset{ 
	language=Matlab,                		% choose the language of the code
%	basicstyle=10pt,       				% the size of the fonts that are used for the code
	numbers=left,                  			% where to put the line-numbers
	numberstyle=\footnotesize,      		% the size of the fonts that are used for the line-numbers
	stepnumber=1,                   			% the step between two line-numbers. If it's 1 each line will be numbered
	numbersep=5pt,                  		% how far the line-numbers are from the code
%	backgroundcolor=\color{white},  	% choose the background color. You must add \usepackage{color}
	showspaces=false,               		% show spaces adding particular underscores
	showstringspaces=false,         		% underline spaces within strings
	showtabs=false,                 			% show tabs within strings adding particular underscores
%	frame=single,	                			% adds a frame around the code
%	tabsize=2,                				% sets default tabsize to 2 spaces
%	captionpos=b,                   			% sets the caption-position to bottom
	breaklines=true,                			% sets automatic line breaking
	breakatwhitespace=false,        		% sets if automatic breaks should only happen at whitespace
	escapeinside={\%*}{*)}          		% if you want to add a comment within your code
}
%%%%%%%%%%%%%%%%


\usepackage[utf8]{inputenc}
\usepackage{float}
\usepackage{subfigure}
\usepackage{mwe}
\usepackage{trfsigns}
\usepackage{amsmath}
\usepackage{xcolor}
\usepackage{bbm}
% \usepackage{hyperref}
\usepackage{hyperref}
\hypersetup{
    bookmarks=true,         % show bookmarks bar?
    unicode=false,          % non-Latin characters in Acrobat’s bookmarks
    pdftoolbar=true,        % show Acrobat’s toolbar?
    pdfmenubar=true,        % show Acrobat’s menu?
    pdffitwindow=false,     % window fit to page when opened
    pdfstartview={FitH},    % fits the width of the page to the window
    pdftitle={My title},    % title
    pdfauthor={Author},     % author
    pdfsubject={Subject},   % subject of the document
    pdfcreator={Creator},   % creator of the document
    pdfproducer={Producer}, % producer of the document
    pdfkeywords={keyword1, key2, key3}, % list of keywords
    pdfnewwindow=true,      % links in new PDF window
    colorlinks=false,       % false: boxed links; true: colored links
    linkcolor=red,          % color of internal links (change box color with linkbordercolor)
    citecolor=green,        % color of links to bibliography
    filecolor=cyan,         % color of file links
    urlcolor=magenta        % color of external links
}

\lstdefinestyle{mlab}{language=Matlab, numbers=left, numberstyle=\tiny,%5
basicstyle={\ttfamily},%
 keywordstyle={\color{blue}},%
 commentstyle=\color{mlgreen},%
 stringstyle=\color{mlviolett},%
 %breaklines=true,
 }
\lstset{frame=tb,
  language=Java,
  aboveskip=3mm,
  belowskip=3mm,
  showstringspaces=false,
  columns=flexible,
  basicstyle={\small\ttfamily},
  %numbers=none,
  numbers=left,
  numberstyle=\tiny,
  keywordstyle=\color{blue},
  commentstyle=\color{dkgreen},
  stringstyle=\color{mauve},
  breaklines=true,
  breakatwhitespace=true,
  tabsize=3
}


\usepackage{mathrsfs}
%\usepackage{subcaption}
\DeclareMathOperator*{\argmax}{arg\,max}
\DeclareMathOperator*{\argmin}{arg\,min}
% Packages used in LNTthesis class:
% package[english]{babel}   % english language / Englische Sprache
% package{LNTthesis}        % LNT specific definitions / LNT spezifische Definitionen
% package{graphicx}         % for using eps images / Einbinden von EPS Grafiken
% package{verbatim}         % for quickly commenting out large parts of your text / Um viel Text schnell auskommentieren zu koennen
% package{amssymb}          % additional math symbols / Zusaetzliche mathematische Symbole
% package{amsmath}          % additional math commands / Zusaetzliche mathematische Befehle
% package{amsxtra}          % even more math symbols / Noch mehr mathematische Symbole
% package{amsthm}           % theorem environment etc / Theorem Umgebung usw
% more information on amsmath: http://www.ctan.org/get/macros/latex/required/amslatex/math/amsldoc.pdf
% package{psfrag}           % psfrag: http://www.ctan.org/get/macros/latex/contrib/psfrag/pfgguide.pdf
% package{subfigure}        % enable subfigures / Ermoeglicht Subfigures (mehrere Figures neben/untereinander)
% !! PLEASE READ THE LATEX HELP IF YOU HAVE ANY QUESTIONS !!
% http://tobi.oetiker.ch/lshort/lshort.pdf
% Macros:
\newcommand{\eq}[1]{Equation (\ref{#1})}        % \eg{eq:golomb}  --> Equation (2.15)
\newcommand{\eref}[1]{(\ref{#1})}               % \eg{eq:golomb}  --> (2.15)
\newcommand{\fig}[1]{Figure \ref{#1}}           % \fig{fig:golomb}--> Figure 2.15
\newcommand{\tab}[1]{Table \ref{#1}}            % \tab{tab:lala}  --> Table 2.15
\newtheorem{prop}{Proposition}
% Abbreviations
\newcommand{\equivalent}{\triangleq}
\newcommand{\given}{\:\!\vert\:\!}
\usepackage{eucal}

\def \A{\mathcal{A}}
\def \B{\mathcal{B}}
\def \C{\mathcal{C}}
\def \D{\mathcal{D}}
\def \E{\mathcal{E}}
\def \F{\mathcal{F}}
\def \G{\mathcal{G}}
\def \H{\mathcal{H}}
\def \I{\mathcal{I}}
\def \J{\mathcal{J}}
\def \K{\mathcal{K}}
\def \L{\mathcal{L}}
\def \M{\mathcal{M}}
\def \N{\mathcal{N}}
\def \O{\mathcal{O}}
\def \P{\mathcal{P}}
\def \Q{\mathcal{Q}}
\def \R{\mathcal{R}}
\def \S{\mathcal{S}}
\def \T{\mathcal{T}}
\def \U{\mathcal{U}}
\def \V{\mathcal{V}}
\def \W{\mathcal{W}}
\def \X{\mathcal{X}}
\def \Y{\mathcal{Y}}
\def \Z{\mathcal{Z}}
\def \sG {\mathscr{G}}
\def \sU {\mathscr{U}}
\def \sT {\mathscr{T}}
\def \fG{\mathbf{G}}
\def \fa{\mathbf{a}}
\def \fb{\mathbf{b}}
\def \fg{\mathbf{g}}
\def \fu{\mathbf{u}}
\def \fv{\mathbf{v}}
\def \fx{\mathbf{x}}
\def \fy{\mathbf{y}}
\def \fz{\mathbf{z}}
\def \fU{\mathbf{U}}
\def \fV{\mathbf{V}}
\def \fX{\mathbf{X}}
\def \fY{\mathbf{Y}}
\def \fZ{\mathbf{Z}}
\def \f0{\mathbf{0}}
\def \tX{\widetilde{X}}
%
\definecolor{blau_1a}{RGB}{93,133,195}
\definecolor{blau_2a}{RGB}{0,156,218}
\definecolor{gruen_3a}{RGB}{80,182,149}
\definecolor{gruen_4a}{RGB}{175,204,80}
\definecolor{gruen_5a}{RGB}{221,223,72}
\definecolor{orange_6a}{RGB}{255,224,92}
\definecolor{orange_7a}{RGB}{248,186,60}
\definecolor{rot_8a}{RGB}{238,122,52}
\definecolor{rot_9a}{RGB}{233,80,62}
\definecolor{lila_10a}{RGB}{201,48,142}
\definecolor{lila_11a}{RGB}{128,69,151}

% Farbpalette B
\definecolor{blau_1b}{RGB}{0,90,169}
\definecolor{blau_2b}{RGB}{0,131,204}
\definecolor{gruen_3b}{RGB}{0,157,129}
\definecolor{gruen_4b}{RGB}{153,192,0}
\definecolor{gruen_5b}{RGB}{201,212,0}
\definecolor{orange_6b}{RGB}{253,202,0}
\definecolor{orange_7b}{RGB}{245,163,0}
\definecolor{rot_8b}{RGB}{236,101,0}
\definecolor{rot_9b}{RGB}{230,0,26}
\definecolor{lila_10b}{RGB}{166,0,132}
\definecolor{lila_11b}{RGB}{114,16,133}

\definecolor{mycolor1}{rgb}{0.0, 0.18, 0.39}
\definecolor{mycolor2}{RGB}{87,108,67}
\definecolor{mycolor3}{RGB}{8,133,161}
\definecolor{mycolor4}{RGB}{80,91,161}
\definecolor{mycolor5}{RGB}{98,122,157}
\definecolor{mycolor6}{RGB}{255,163,67}
\definecolor{mycolor7}{RGB}{152,205,225}
\definecolor{mycolor8}{RGB}{242,204,48}
\definecolor{mycolor9}{rgb}{0,.5,0}
\definecolor{mycolor10}{rgb}{.59,.44,.09}
%
\definecolor{mycolor11}{RGB}{231,199,31} % Yellow
\definecolor{mycolor12}{RGB}{8,133,161} % Cyan
\definecolor{mycolor13}{RGB}{157,188,64} % Yellow Green
\definecolor{mycolor14}{RGB}{194,150,130} % Light Skin
\definecolor{mycolor15}{RGB}{98,122,157} % Blue Sky
\definecolor{mycolor16}{RGB}{160,160,160} % Neutral
\definecolor{mycolor17}{RGB}{115,82,68} % Dark Skin
\definecolor{mycolor18}{RGB}{94,60,108} % Purple
\definecolor{mycolor19}{RGB}{115,82,68} % Dark Skin
\definecolor{mycolor20}{RGB}{255,183,30} % Dark Gold

\definecolor{mlgreen}{rgb}{.035,.6,.251}
\definecolor{mlviolett}{rgb}{.643,.259,.804}
\definecolor{dkgreen}{rgb}{0,0.6,0}
\definecolor{gray}{rgb}{0.5,0.5,0.5}
\definecolor{mauve}{rgb}{0.58,0,0.82}

\theoremstyle{remark}	\newtheorem{theorem}{Theorem}
\theoremstyle{remark}	\newtheorem{lemma}[theorem]{Lemma}
\theoremstyle{remark}	\newtheorem{corollary}[theorem]{Corollary}
\theoremstyle{remark}	\newtheorem{proposition}[theorem]{Proposition}
\theoremstyle{remark} \newtheorem{definition}{Definition}
\theoremstyle{remark} \newtheorem{remark}{Remark}
\theoremstyle{remark} \newtheorem{example}{Example}
%
\newcommand{\ceil}[1]{\lceil #1 \rceil}
\newcommand{\bigceil}[1]{\Bigl \lceil #1 \Bigr \rceil}
\newcommand{\floor}[1]{\lfloor #1 \rfloor}
\newcommand{\bigfloor}[1]{\Bigl \lfloor #1 \Bigr \rfloor}
%
\usepackage{tikz}
\usepackage{circuitikz}
\usepackage{pgfplots} 
\usepackage{amsmath,amssymb,amsfonts}
\usepackage{commath}
\usepackage{eucal}
\usepackage{mathtools}
\usepackage{amsthm}
\usepackage{algorithmic}
\usepackage{graphicx}
\pgfplotsset{compat=1.15}
\usepackage{graphicx}
\usepackage{capt-of}
\usepackage{lipsum}
\usepackage{float}
\usepackage{pgfgantt}

% \renewcommand{\epsilon}{\varepsilon}
\usepackage{diffcoeff}

\usepackage{autobreak}
\allowdisplaybreaks

% \documentclass{amsart}
\usepackage{mathtools}
\usepackage{tikz}
\usetikzlibrary{positioning}


\usepackage{silence}
\WarningFilter{ctable}{Transparency disabled:}
\usepackage{ctable}

\usepackage{pgfplots}




\usepackage{amsbsy}
\usepackage{bm}
\usepackage{fixmath}

\usepackage{silence}
\WarningFilter{ctable}{Transparency disabled:}
\usepackage{ctable}

\usepackage{pst-node,pst-plot}

\usetikzlibrary{shapes.geometric}

\usepackage{mathdots}

% \usepackage{pgf,tikz}

\usetikzlibrary{math}